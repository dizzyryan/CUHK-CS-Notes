\chapter{Set Theory and Counting Principle}

\section{Set Theory}

Some concepts of Set Theory will be introduced in this section.

Set is simply a collection of elements, in which each element appears only once. 
For example, let \(S\) be the set of first 5 positive integers.
\[
    S = \{1, 2, 3, 4, 5\}
\]

Given a set \(S\), the number of elements in \(S\) is denoted by \(\vert S \vert \). The quantity \(\vert S \vert \) is also referred to as the \textbf{cardinality} of \(S\).

For example, the above set has cardinality \(\vert S \vert = 5\). 

A set with no elements in it is called an empty set, which is denoted by \(\varnothing\).

Given two sets \(S\) and \(T\), we say that \(T\) is a subset of \(S\), denoted by \(T \subseteq S\), if every element in \(T\) is also in \(S\). It follows that if \(T \subseteq S\), then \(\vert T \vert \leq \vert S \vert \). 

For example, let
\[
    S = \{1, 2, 3, 4, 5\},\quad T = \{2, 4\}, \quad U = \{2, 4, 6\}
\]

Then, we have \(T \subseteq S\) and \(T \subseteq U\).

Let \(S_1\) and \(S_2\) be two given sets. 

The union of \(S_1\) and \(S_2\), denoted by \(S_1 \cup  S_2\), is the set containing all elements from both \(S_1\) and \(S_2\). 

We can also see union as the relationship "or", while intersection is the relationship "and".

Now, let \(S\) be a set and \(m \geq 1\) be an integer. A partition of \(S\) into \(m\) parts is a collection of \(m\) subsets of \(S\), denoted by \(S_1, \dots, S_m\), with the following properties:
\begin{itemize}
    \item Exhaustion: \(S = S_1 \cup S_2 \cup \cdots \cup S_m.\)
    \item Non-Overlapping: \(\text{For}\ i \neq j\), we have \(S_i \cap S_j = \varnothing\)
\end{itemize}

\section{Counting Principle}
\subsection{Addition Principle}
Again we use the above set 
\[
    S = \{1, 2, 3, 4, 5\}
\]
as an example. Let \(S_1 = \{1, 2\}, S_2 = \{3\}, S_3 = \{4, 5\}\) form a partition of \(S\) with 3 parts. Indeed, we have \(S = S_1 \cup S_2 \cup S_3\) and \(S_1 \cap S_2 = \varnothing\), \(S_1 \cap S_3 = \varnothing\), \(S_2 \cap S_3 = \varnothing\).
It also holds true that if we can break \(S\) into non-overlapping parts, the cardinality of \(S\) can be determined by the cardinalities of the constituent parts, i.e. \(\vert S \vert = \vert S_1 \vert + \vert S_2 \vert + \cdots + \vert S_m \vert \).

\begin{eg}
    How to count the number of binary strings of length \(n\) with no consecutive 1's? Let \(S\) be the desired string, then we can apply the addition principle.

    Let \(S_0\) be the set of strings in \(S\) that start with 0; \(S_1\) be the set of strings in \(S\) that start with 1. We have \(\vert S \vert = \vert S_0 \vert + \vert S_1 \vert\).

    We can set up a recurrence relationship. Let \(\vert S \vert = T(n)\). Then, we have \(\vert S_0 \vert = T(n - 1)\) and \(\vert S_1 \vert = T(n - 2)\).

    Then we have 
    \[
        \vert S \vert = T(n) = T(n - 1) + T(n - 2)
    \]

    Using the initial conditions \(T(1) = 2\) (string starts with 1 or 0) and \(T(2) = 3\) (if the string starts with 1, then we have only one option for the next digit; otherwise, we have two options), one can find the number of the desired binary strings. 
\end{eg}

\section{Multiplication Principle}
Multiplication principle states that if each element in \(S\) can be generated by performing an ordered sequence of actions, say, \(A_1, A_2, \dots, A_N\), and action \(A_i\) has \(p_i\) choices, where \(i = 1, \dots, N\), then the cardinality of \(S\) can be computed by
\[
    \vert S \vert = p_1 \times p_2 \times \cdots \times p_N
\] 

\begin{eg}
    Consider a set of \(n\) elements: \(S_0 = \{1, \dots, n\}\). We call \(S_0\) the ground set. Also, let \(r \geq 1\) be integer. An \(r\)-permutation of the \(n\)-elements ground set \(S_0\) is an ordered selection of \(r\) elements from \(S_0\). Let \(S\) be the set of all different \(r\)-permutations of the \(n\)-element ground set \(S_0\). For instance, when \(S_0 = \{1, 2, 3\}\) and \(r = 2\), we have
    \[
        S = \{(1, 2), (2, 1), (1, 3), (3, 1), (2, 3), (3, 2)\},
    \]
    where the order of the two elements matters. 
\end{eg}

For general values of \(n\) and \(r\). Let \(P(n, r)\) denotes this number. By performing the following ordered sequence of action,
\begin{table}[H]
    \centering
    \begin{tabular}{c|c}
            action & number of choices  \\
        \midrule
            \(A_1\): pick the 1st element & \(n\)   \\
            \(A_2\): pick the 2nd element & \(n - 1\)   \\
            \vdots & \vdots  \\
            \(A_r\): pick the \(r\)th element & \(n - r + 1\)
    \end{tabular}
\end{table}
Thus, we have
\[
    P(n, r) = n(n - 1)\cdots(n - r + 1) = \dfrac{n!}{(n - r)!}
\]

\begin{eg}
    Let us count the number of integers between 0 and 9999 that have exactly one digit equal to 5. 

    Let \(S\) be the set of such integers. We can then partition this set to \(S_1, S_2, S_3, S_4\),  where \(S_1\) is the set of integers in \(S\) with "5" appearing in the first position from the right, and so on. For example, we have \(2051 \in S_2\). 

    By the addition principle, we have
    \[
        \vert S \vert = \vert S_1 \vert + \vert S_2 \vert + \vert S_3 \vert + \vert S_4 \vert 
    \]

    Since every integer in \(S_1\) takes the form \(xxx5\), with each "\(x\)" having 9 choices. We then have \(\vert S_1 \vert = 9^3\). It holds true for the other sets. Hence, we have \(\vert S \vert = 4 * 9^3 = 2916\)
\end{eg}
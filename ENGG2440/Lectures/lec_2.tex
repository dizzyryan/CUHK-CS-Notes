\chapter{Summation Techniques}

\section{Summation}
When summing numbers with certain patterns, we can use summation notation. For example,
\[
	a_1 + a_2 + \cdots + a_n = \sum_{k=1}^n a_k 
\]

\subsection{Distributive Law}
Let \(c\) be a constant. Then, we can take \(c\) out of the summation:
\[
	\sum_{k\in\mathcal{K} } ca_k = c\sum_{k\in\mathcal{K} }a_k 
\]

\begin{eg}
	\[
		\sum_{k=1}^n 2k = 2(1) + 2(2) + 2(3) + \cdots + 2(n) = 2(1 + 2 + 3 + \cdots + n) = 2\sum_{k=1}^n k  
	\]
\end{eg}

\subsection{Associative Law}
We can split the summands as follows:

\[
	\sum_{k\in\mathcal{K}}(a_k + b_K) = \sum_{k\in\mathcal{K}}a_k + \sum_{k\in\mathcal{K}}b_k
\]

\begin{eg}
	\[\begin{aligned}
		\sum_{k = 1}^n(k + k^2) &= (1 + 1^2) + (2 + 2^2) + \cdots + (n + n^2) \\
		&= (1 + 2 + \cdots + n) + (1^2 + 2^2 + \cdots + n^2) \\
		&= \sum_{k = 1}^n k + \sum_{k = 1}^n k^2
	\end{aligned}\]
\end{eg}

\section{Close Form Formula}
\chapter{Asymptotics}

Asymptotic notation is a shorthand used to give a quick measure of the behavior of a function \(f(n)\) as \(n\) grows large. 

\section{Big O}
Big O is the most frequently used asymptotic notation. It is used to give an upper bound on the growth of a function, such as the running time of an algorithm.

\begin{definition}
    We say that \(f(x) = O(g(x))\) iff there exists a constant \(c > 0\) and an \(x_0 \geq 0\) such that
    \[
    f(x) \leq cg(x)\quad\text{for all}\ x \geq x_0
    \]
\end{definition}

Let's see an example
\begin{eg}
    Let \(f(x) = x^2\) and \(g(x) = x^3\). 

    By taking \(c = 1\) and \(x_0 = 1\), we can simply conclude that \(x^2 = O(x^3)\). 
    
    However, to prove that \textbf{there is no constant \(c\) such that \(x^2 \leq cx^3\) for all \(0 \leq x \leq 1\)}, we need to use proof by contradiction. 
    
    For \(0 \leq x \leq 1\), we have \(x^2 \geq x^3\). Therefore, if such a constant \(c > 0\) exists, then we must have \(c \geq 1\) such that \(x^2 \leq cx^3\). However, whenever \(0 \leq x \leq \frac{1}{c}\), we have
    \[
        x^3 \geq \dfrac{1}{c}x^2 > x^3.
    \]
    Since we get \(x^3 > x^3\), by contradiction, we know that there is no constant \(c\) such that \(x^2 \leq cx^3\) for all \(0 \leq x \leq 1\)
\end{eg}

We can also use differentiation to show the upper bound of a function.
\begin{eg}
Let \(f(x) = \ln x\) and \(g(x) = x\). 

Observe that \(f(1) = \ln 1 = 0 < 1 = g(1)\).  Moreover, \(f^{\prime} (x) = \frac{1}{x}\) and \(g^{\prime} (x) = 1\) for all \(x > 0\), which implies that \(g^{\prime}(x) > f^{\prime}(x)\) for all \(x > 1\). It is the same as saying that \(g(x)\) grows faster than the function \(f(x)\) because the slope of the former is larger than that of the latter. It follows that
\[
    \ln x \leq x\ \text{for all}\ x \geq 1.
\] 
Hence, by taking \(c = 1\) and \(x_0 = 1\) we have \(\ln x = O(x).\) 
\end{eg}

\begin{exercise}
    Let \(f(x) = x^2\) and \(g(x) = 2^x\). Show that \(f(x) = O(g(x))\).
\end{exercise}

\begin{eg}
    Let
    \[
        f(n) = n \cdot (n + 1) \cdot (n + 2) + (-1)^n, \quad g(n) = n^\alpha 
    \]
    for any integer \(n \geq 1\). What is the smallest \(\alpha \in \mathbb{R}\) such that \(f(n) = O(g(n))\)?
    Note that
    \[
        \begin{aligned}
            \vert f(n) \vert &= \vert n \cdot (n + 1) \cdot (n + 2) + (-1)^n \vert \\
            &= \vert n^3 + 3n^2 + 2n + (-1)^n \vert \\
            &= n^3 + 3n^2 + 2n + (-1)^n\quad (\text{for any}\ n \geq 1)
        \end{aligned}
    \] 
    We claim that \(\alpha = 3\) is the smallest \(\alpha \in \mathbb{R}\) such that \(f(n) = O(g(n))\).

    Let us first prove that \(\alpha\) satisfies \(\alpha \geq 3\) and then that \(\alpha = 3\). 

    Indeed, we must have 
\end{eg}


\section{Big Omega}
As we use Big O notation to express upper bound, for lower bound, we have the "Big Omega" notation.
\begin{definition}
    We say that \(f(x) = \Omega(g(x))\) iff there exists a constant \(c > 0\) and an \(x_0 \geq 0\) such that
    \[
    f(x) \geq cg(x)\quad\text{for all}\ x \geq x_0
    \]
\end{definition}

Since Big O and Big Omega are essentially "mirror image" of one another, we have

\begin{theorem}[Big O vs. Big Omega]
   \[
    f(x) = O(g(x)) \Longleftrightarrow g(x) = \Omega (f(x)).
   \] 
\end{theorem}

\begin{proof}
    By definition of big O, \(f(x) = O(g(x))\) means
    \[
        \exists c_1 > 0, x_1 > 0 \text{ such that } f(x) \leq c_1 g(x) \quad \forall x \geq x_1,
    \] 
    which means the same as
    \[
        \exists c_1 > 0, x_1 > 0 \text{ such that } g(x) \geq \dfrac{1}{c_1}f(x) \quad \forall x \geq x_1,
    \]
    Hence, by taking \(c_0 = \frac{1}{c_1} > 0\) and \(x_0 = x_1 \geq 0\), we have \(g(x) = \Omega(f(x))\) 
\end{proof}

\section{Theta}
Theta can be understood as the approximation of a function. 
\begin{definition}
    We say that \(f(x) = \Theta(g(x))\) iff \(f(x) = O(g(x))\) and \(g(x) = O(f(x))\). 
    \begin{remark}
        It is important to note that \(f(x) = \Theta(g(x))\) does not mean that \(f(x) = g(x)\), it just means that
    \[
        \exists c_1, _2 > 0 \quad\text{such that}\quad c_1 g(x) \leq f(x) \leq c_2 g(x)\quad \forall x \geq  x_0 
    \] 
    \end{remark}
    
\end{definition}

\section{Little O}
Little O notation can be understood as the strict upper bound on the growth of a function.

\begin{definition}[Little O notation]
    For functions \(f, g: \mathbb{R} \to \mathbb{R}\), with \(g\) nonnegative, we say \(f\) is asymptotically smaller than \(g\), in symbols,
    \[
        f(x) = o(g(x))\quad\text{iff}\quad\lim_{x \to \infty} \dfrac{f(x)}{g(x)} = 0. 
    \]
\end{definition}

\begin{eg}
    For example, let \(f(x) = x, g(x) = e^x - 1\). Because 
    \[
        \lim_{x \to \infty} \dfrac{f(x)}{g(x)} = \lim_{x \to \infty} \dfrac{x}{e^x - 1} = 0.
    \]
    Hence, we have \(f(x) = o(g(x))\) 
\end{eg}
\newpage
\begin{eg}
Let
\[
    f(n) = \begin{dcases}
        0\quad \text{if \(n\) is odd},\\
        2\quad \text{if \(n\) is even}.
    \end{dcases}
\]
Is \(f(n) = o(1)\)?

Let \(g(n) = 1\). Upon noting \(f(n) = 1 + (-1)^n\), we see that
\[
    \lim_{n \to \infty} \dfrac{f(x)}{g(x)} = \lim_{n \to \infty} f(n).
\] 
However, this limit does not exist, as \(f(n)\) fluctuates between 0 and 2. Thus, \(f(n) \neq o(1)\).

On the other hand, if \(f(n) = o(n)\)? Here, let \(g(n) = n\). Note that \(0 \leq f(n) \leq 2\). It follows that 
\[
    0 \leq \dfrac{f(n)}{g(n)} \leq \dfrac{2}{n}
\]
for \(n \geq 1\). By the sandwich theorem, we then obtain
\[
    \lim_{n \to \infty} \dfrac{f(n)}{g(n)} = 0.
\] 
Thus, \(f(n) = o(n)\).
\end{eg}

\section{Little Omega}
Little Omega notation can be understood as the strict lower bound on the growth of a function.

\begin{definition}[Little O notation]
    For functions \(f, g: \mathbb{R} \to \mathbb{R}\), with \(g\) nonnegative, we say \(f\) is asymptotically smaller than \(g\), in symbols,
    \[
        f(x) = \omega (g(x))\quad\text{iff}\quad\lim_{x \to \infty} \dfrac{g(x)}{f(x)} = 0. 
    \]
\end{definition}

\section{Properties for Asymptotic Analysis}
\subsection{Rules for Asymptotic Analysis}
\begin{itemize}
    \item Transitivity
        \[
            \text{If}\ f(n) = \Pi (g(n))\ \text{and}\ g(n) = \Pi (h(n)),\ \text{then}\ f(n) = \Pi (h(n)),\ \text{where}\ \Pi = O, o, \Omega, \omega, \Theta
        \]
    \item Rule of sums
        \[
            f(n) + g(n) = \Pi (max\{f(n), g(n)\}),\ \text{where}\ \Pi = O, \Omega,\ \text{or}\ \Theta.
        \]
    \item Rule of products
        \[
            \text{If}\ f_1(n) = \Pi (g_1(n)), f_2(n) = \Pi (g_2(n)),\ \text{then}\ f_1(n)f_2(n) = \Pi (g_1(n)g_2(n)),\ \text{where}\ \Pi = O, o, \Omega, \omega, \Theta.
        \]
    \item Transpose symmetry
        \[
            f(n) = O(g(n))\ \text{iff}\ g(n) = \Omega(f(n)).
        \]
    \item Transpose symmetry
        \[
            f(n) = o(g(n))\ \text{iff}\ g(n) = \omega(f(n)).
        \]
    \item Reflexivity
        \[
            f(n) = \Pi(f(n)),\ \text{where}\ \Pi = O, \Omega, \Theta.
        \]
    \item Symmetry
        \[
            f(n) = \Theta(g(n))\ \text{iff}\ g(n) = \Theta(f(n)).
        \]
\end{itemize}

\subsection{Graph for Functions}
One can understand the growth of functions by the following graph.

\begin{center}
    \import{./Figures/}{time_complexity.tex}
\end{center}

% END OF DOCUMENT
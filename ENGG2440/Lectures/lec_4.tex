\chapter{Asymptotics}

Asymptotic notation is a shorthand used to give a quick measure of the behavior of a function \(f(n)\) as \(n\) grows large. For example, "little oh" indicating that one function grows at
a significantly slower rate than another, and "Big Oh" indicating that one function grows not much more rapidly than another.

\section{Little O}

\begin{definition}[Little O notation]
    For functions \(f, g\ :\ \mathbb{R} \to \mathbb{R}\), with \(g\) nonnegative, we say \(f\) is asymptotically smaller than \(g\), in symbols,
    \[
        f(x) = o(g(x))\quad\text{iff}\quad\lim_{x \to \infty} \dfrac{f(x)}{g(x)} = 0. 
    \]
\end{definition}
For example, \(x = o(e^x - 1)\), because 
\[
    \lim_{x \to \infty} \dfrac{x}{e^x - 1} = 0.
\]

By observing another example, \(1000x^{1.9} = o(x^2)\), since \(1000x^{1.9} / x^2 = 1000 / x^0.1\), \(x^{0.1}\) goes to infinity with x and 1000 is constant. Then the limit will equal 0. This argument generalizes directly to yield
\begin{lemma}
    \(x^a = o(x^b)\) for all nonnegative constants \(a < b\).
\end{lemma}

Since \(\log x < x\) for all \(x > 1\), 
\begin{lemma}
    \(\log x = o(x^\epsilon)\) for all \(\epsilon > 0\)
\end{lemma}

\begin{corollary}
    
\end{corollary}
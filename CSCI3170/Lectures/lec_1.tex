\chapter{Introduction}

\section{Overview}
Data will always need to be stored, manipulated, accessed, shared, and transmitted. Thus, we require certain methods to handle it. 

A data table (or data frame) is a two-dimensional structure. Regarding the data itself, we generally categorize it into three types. The first is \textbf{Unstructured Data}, which refers to information that does not follow a specific format, such as the statement: ``a university has 10,000 students.'' Next, we have \textbf{Semi-structured Data}, which has some organizational elements (e.g., tags, hierarchies) but is still difficult to process directly. Finally, we have \textbf{Structured Data}, the most organized type, stored in predefined formats such as tables with rows and columns. 

To manage such vast amounts of data, we use \textbf{Database Management Systems (DBMS)}, which are software packages designed to maintain and utilize large collections of data. 

By using a DBMS to store data, we ensure data independence, data integrity, security, concurrent access, and crash recovery.

First, we begin with the \textbf{conceptual model}.
\chapter{Probability Models and Axioms}

\section{Basic Definitions}
We will introduce some definitions here as well. 

\begin{definition}[Complement]
    The complement of event \(A\) (denoted by \(A^c\)) is the opposite event of \(A\). In other words, \(A^c\) happens if and only if \(A\) does not happen.
\end{definition}

Again, when flipping three coins, we have the following sample space:
\[
    \Omega = \{\text{HHH}, \text{HHT}, \text{HTH}, \text{HTT}, \text{THH}, \text{THT}, \text{TTH}, \text{TTT}\}.
\]
Let \(A\) be the event that at least two heads occur. Then for \(A^c\), we have:
\[
    A^c = \{\text{TTT}, \text{HTT}, \text{THT}, \text{TTH}\}
\]

\begin{definition}[Intersection of Events]
    The intersection of events happens when all the events occur. We denote this intersection of event \(A\) and \(B\) with \(A \cap B\). 
\end{definition}

Let \(B\) be the event that no consecutive heads occurs. Then, for \(A \cap B\), we have the event that at least two heads and no consecutive heads occur.
\[
    A \cap B = \{\text{HTH}\}
\]

\begin{definition}[Union of Events]
    The union of events happens when at least one of the events occur. We denote the union of events \(A\) and \(B\) with \(A \cup B\). 
\end{definition}

For example, for \(A \cup B\) in the above example, we have 
\[
    A \cup B = \{\text{HHH}, \text{HHT}, \text{HTH}, \text{HTT}, \text{THH}, \text{THT}, \text{TTH}, \text{TTT}\}
\]

\begin{definition}[Disjoint Events]
    We call event \(A_1, A_2, \cdots\) disjoint events (or mutually exclusive events) if the intersection of every two events \(A_i, A_j (i \neq j)\) is the null event:
    \[
        \forall i \neq j:\quad A_i \cap A_j = \varnothing
    \]  
\end{definition}

Let \(C\) be the event that at least three heads occur. Then 
\[
    B \cap C = \varnothing.
\]
\newpage

\section{Probability Axioms}

\begin{definition}[Axioms of Probability]
    A probability assignment \(\mathcal{P}\) to sample space \(\Omega\) should satisfy the following three axioms:
    \begin{enumerate}
        \item For every event \(A\), \(0 \leq \mathbb{P}(A)\); 
        \item \(\mathbb{P}(\Omega) = 1\); 
        \item If event \(A_1, A_2, \cdots\) are disjoint, \(\mathbb{P}(A_1 \cup A_2 \cup \cdots) = \mathbb{P}(A_1) + \mathbb{P}(A_2) + \cdots\)
    \end{enumerate}
\end{definition}

Follow these axioms, and we can prove most of the rules for probability calculation.

\section{Rules for Probability Calculation}

\begin{proposition}[Complement Rule]
    For every event \(E\) and its complement \(E^c\):
    \[
        \mathbb{P}(E^c) = 1 - \mathbb{P}(E)
    \]
\end{proposition}

\begin{proposition}[Difference Rule]
    If event \(E\), \(F\) satisfy \(E \subseteq F\), then: 
    \[
        \mathbb{P}(F \cap E^c) = \mathbb{P}(F) - \mathbb{P}(E)
    \]

    \begin{remark}
        As a result, if \(E \subseteq F\), then \(\mathbb{P}(E) \leq \mathbb{P}(F)\) 
    \end{remark}
    \begin{proof}
        \[
            \begin{aligned}
                \mathbb{P}(F \cap E^c) &= \mathbb{P}(F) - \mathbb{P}(E) \\
                \mathbb{P}(F) &= \mathbb{P}(F \cap E^c) + \mathbb{P}(E) \\
            \end{aligned}
        \]
        Since \((F \cap E^c) \cap E = F \cap (E^c \cap E) = F \cap \varnothing = \varnothing\), \(\mathbb{P}(F \cap E^c) + \mathbb{P}(E) \Rightarrow (F \cap E^c) \cup E\)
        \[
            \begin{aligned}
                (F \cap E^c) \cup E &= (F \cup E) \cap (E^c \cup E) \\
                (F \cap E^c) \cup E &= (F \cup E) \cap \Omega \\
                (F \cap E^c) \cup E &= F \cup E \\
                (F \cap E^c) \cup E &= F\ (\text{for}\ E \subseteq F)\\
            \end{aligned}
        \]
    \end{proof}
\end{proposition}

\begin{proposition}[Inclusion-Exclusion Principle]
    For events \(E, F\):
    \[
        \mathbb{P}(E \cup F) = \mathbb{P}(E) + \mathbb{P}(F) - \mathbb{P}(E \cap F)
    \]

    \begin{remark}
        We can generalize the principle to more than two events. For example, 
        \[
            \mathbb{P}(E \cup F \cup G) = \mathbb{P}(E) + \mathbb{P}(F) + \mathbb{P}(G) - \mathbb{P}(E \cap F) - \mathbb{P}(E \cap G) - \mathbb{P}(F \cap G) + \mathbb{P}(E \cap F \cap G)
        \]
    \end{remark}
\end{proposition}

\begin{eg}
In a city, \(10\%\) of the people are rich, \(5\%\) are famous, and \(3\%\) are both rich and famous. For a randomly-selected person in the city, find the probability for the following Events. 

Here we let \(R\) be the event that the person is rich, \(F\) be the event that the person is famous, 

1. The person is not rich.
\[
    \mathbb{P}(R^c) = 1 - \mathbb{P}(R) = 1 - 0.1 = 0.9
\]

2. The person is not rich but is famous.
\[
    \mathbb{P}(R^c \cap F) = \mathbb{P}(F) - \mathbb{P}(F \cap R) = 0.05 - 0.03 = 0.02
\]

3. The person is neither rich nor famous.
\[
    \mathbb{P}(F^c \cap R^c) = 1 - \mathbb{P}(F \cup R) = 1 - \mathbb{P}(F) - \mathbb{P}(R) + \mathbb{P}(F \cap R) = 1 - 0.05 - 0.1 + 0.03 = 0.88
\]
\end{eg}

% END OF DOCUMENT
\chapter{Recurrences}

Sometimes recurrences are closely related to sums. Thus, if we are going to find the closed-form formula, we can express the sum as a recurrence, then the problem will be comparatively easier.

\section{Homogeneous Recurrences}
\begin{definition}[Linear homogeneous recurrence]
    A linear homogeneous recurrence relation of degree \(d\) with constant coefficients is a recurrence relation of the form
    \[
        T(n) = a_1 T(n-1) + a_2 T(n-2) + \cdots + a_d T(n-d) ,
    \]
    where \(a_1, a_2, \cdots, a_k \in \mathbb{R}\) are given constants and \(a_k \neq  0\) 
\end{definition}

To solve for homogeneous recurrences with distinct root, we can use the following procedure:

1. Solve the characteristic equation to get root \(r_1, \cdots, r_d\).

2. If the roots are all distinct, form, the candidate solution
\[
    T_0(n) = \theta_1 r_1^n + \theta_2 r_2^n + \cdots + \theta_d r_d^n
\]

3. Use the initial conditions on \(T(1), \dots, T(d)\) to determine \(\theta_1, \dots, \theta_d\) 

\begin{eg}
    \[
        \begin{dcases}
            T(n) = T(n - 1) + 2T(n - 2)\quad\text{for \(n \geq  2\) }\\
            T(0) = 2, T(1) = 7
        \end{dcases}
    \]
    Let \(T(n) = x^n\). Then, for characteristic equation, we have
    \[
    \begin{aligned}
        x^n &= x^{n-1} + 2x^{n-2} \\
        x^2 &= x + 2 \quad \text{(characteristic equation)}
    \end{aligned}
    \]
    The roots are \(r_1 = 2, r_2 = -1\). Since they are distinct, we can form the candidate solution
    \[
        T_0(n) = \theta_1 2^n + \theta_2 (-1)^n.
    \]
    By using the initial conditions, we have
    \[
        \begin{aligned}
            2 &= T_0(0) = \theta_1 + \theta_2 \\
            7 &= T_0(1) = 2\theta_1 - \theta_2
        \end{aligned}
    \]
    Solving above, we have \(\theta_1 = 3, \theta_ = -1\). Then, we have
    \[
        T(n) = 3 \times 2^n - (-1)^n
    \] 
\end{eg}

However, if the root of the characteristic equation has a multiplicity \(m \geq 1\), i.e., the root is repeated for \(m\)  times, then we have \(T(n) = n^{m-1}x^n\). Yet the procedures are the same as solving linear homogeneous recurrence with distinct roots. 

\begin{eg}
    \[
        \begin{dcases}
            T(n) = 2T(n - 1) - T(n - 2)\quad\text{for} n \geq 2,\\
            T(0) = 0, T(1) = 1\\
        \end{dcases}
    \]
    Characteristic equation: \(x^2 = 2x - 1\). 

    Solving above we have \(r_1 = 1\) with multiplicity \(m_1 = 2\).
    Then we have
    \[
        T_0(n) = \theta_1 (1)^n + n\theta_2(1)^n = \theta_1+ n\theta_2
    \]
    By using initial conditions, we have
    \[
        \begin{aligned}
            0 &= T_0(0) = \theta \\
            1 &= T_0(1) = \theta_1 + \theta_2
        \end{aligned}
    \]
    Then, it follows that the solution to the recurrences is given by
    \[
        T(n) = n
    \]
\end{eg}

Let's see another example
\begin{eg}
\[
    \begin{dcases}
        T(n) = 4T(n-1) - 5T(n-2) + 2T(n-3)\quad\text{for} n\geq 3,\\
        T(0) = 0, T(1) = 1, T(2) = 3.
    \end{dcases}
\]
Characteristic equation: \(x^3 = 4x^2 - 5x + 2\) 

This equation has two distinct roots, \(r_1 = 1, r_2 = 2\). The multiplicity of \(r_1\) is \(m_1 = 2\). Hence,
\[
    T_0(n) = \theta_1 (1)^n + n\theta_2 (1)^n + \theta_3 2^n = \theta_1 + n\theta_2 + \theta_3 2^n
\]
Using initial conditions, we have
\[
    \begin{aligned}
        0 &= T_0(0) = \theta_1 + \theta_3 \\
        1 &= T_0(1) = \theta_1 + \theta_2 + 2\theta_3 \\
        3 &= T_0(2) = \theta_1 + 2\theta_2 + 4\theta_3
    \end{aligned}
\]
Solving for above, we have
\[
    T(n) = -1 + n\times0 + 1 \times 2^n = -1 + 2^n
\]
\end{eg}

\section{Non-homogeneous Recurrences}
A recurrence relation of the form
\[
    T(n) = a_1 T(n-1) + a_2 T(n-2) + \cdots + a_d T(n-d) + f(n)
\]
is called non-homogeneous recurrences. 

To solve for non-homogeneous recurrences, we can use the following procedures:

1. Solve the associated linear homogeneous recurrence.

2. Find the particular solution \(T_p(n)\) to the linear non-homogeneous recurrence by examining the function class of \(f(n)\). 

3. Form the candidate solution \(T_0(n) = T_h(n) + T_p(n)\), and use the initial conditions to find the parameters in \(T_h(n)\).  

In general, to solve non-homogeneous recurrence, we can consider the following particular solutions:
\begin{table}[H]
    \centering
    \begin{tabular}{c|c}
        \toprule
            \(f(n)\)  & \(T_p(n)\)  \\
        \midrule
            \(s\) & \(x_0\)  \\
            \(n\) & \(x_1n + x_0\)  \\
            \(n^2\) & \(x_2n^2 + x_1n + x_0\)  \\
            \(s^n\) & \(x_0 s^n\)  \\
            \(ns^n\) & \((x_1n + x_0)s^n\) \\
        \bottomrule
    \end{tabular}
\end{table}

Let's see some examples
\begin{eg}
    Consider the recurrence
    \[
    \begin{dcases}
        T(n) = 2T(n-1) + 1\quad\text{for}\ n \geq 1\\
        T(1) = 1
    \end{dcases}
    \]
    Let \(T_0(n) = T_h(n) + T_p(n)\), where \(T_h(n) = 2T(n - 1)\). 

    For \(T_h(n)\), characteristic equation: \(x = 2\), then we have \(T_h(n) = \theta 2^n\).

    Since \(f(n) = 1\), let \(T_p(n) = x\),
    \[
        \begin{aligned}
            x &= 2x + 1 \\
            x &= -1
        \end{aligned}
    \]
    Then, we have
    \[
        T_0(n) = T_h(n) + T_p(n) = \theta 2^n - 1
    \]
    Using the initial conditions, we have
    \[
        \begin{aligned}
            1 &= T_0(1) = 2\theta - 1
        \end{aligned}
    \]
    This gives \(\theta = 1\). Hence, the solution to the recurrence is given by
    \[
        T(n) = 2^n - 1
    \]
\end{eg}


\begin{eg}
    Consider the recurrence
    \[
        \begin{dcases}
            T(n) = 5T(n - 1) - 6T(n - 2) + 7^n\quad\text{for} n \geq 2\\
            T(0) = 0, T(1) = 1.
        \end{dcases}
    \]
    Let \(T_0(n) = T_h(n) + T_p(n)\), where \(T_h(n) = 5T(n - 1) - 6T(n - 2)\). 
    
    For \(T_h(n)\), characteristic equation: \(x^2 = 5x - 6\), with \(r_1 = 3, r_2 = 2\). 
    
    Hence, we have \(T_h(n) = \theta_1 3^n + \theta_2 2^n\), which is the homogeneous solution. 

    Since \(f(n) = 7^n\) and \(s = 7\) is not a root of the characteristic equation. Let \(T_p(n) = x_0 7^n\) (particular solution),
    \[
        \begin{aligned}
            x_0 7^n &= 5x_0 7^{n-1}  - 6x_0 7^{n-2} + 7^n \\
            x_0 &= 5x_0 7^{-1}  - 6x_0 7^{-2} + 1 \\
            x_0 - \dfrac{5}{7}x_0 + \dfrac{6}{49}x_0 &= 1 \\
            x_0 &= \dfrac{49}{20}
        \end{aligned}
    \]
    Then, we have
    \[
        T_0(n) = T_h(n) + T_p(n) = \theta_1 3^n + \theta_2 2^n + \dfrac{49}{20} 7^n
    \]
    Using the initial conditions, we have
    \[
        \begin{aligned}
            0 &= T_0(0) = \theta_1 + \theta_2 + \dfrac{49}{20} \\
            1 &= T_0(1) = 3\theta_1 + 2\theta_2 + \dfrac{343}{20}
        \end{aligned}
    \]
    This gives \(\theta_1 = -\dfrac{225}{20}, \theta_2 = \dfrac{176}{20}\). Hence, the solution to the recurrence is given by
    \[
        T(n) = -\dfrac{225}{20} 3^n + \dfrac{176}{20} 2^n + \dfrac{49}{20} 7^n
    \]
\end{eg}



\begin{remark}
    Let \(f(n) = s^n\), \(r_1\) and \(r_2\) be the roots of the characteristic equation. Then
    \begin{itemize}
        \item If \(s \neq r_1\), \(s \neq r_2\), then \(T_p(n) = x_0 s^n\);
        \item If \(s = r_1\), \(r_1 \neq r_2\), then \(T_p(n) = x_0 ns^n\);
        \item If \(s = r_1 = r_2\), then \(T_p(n) = x_0 n^2 s^n\);
    \end{itemize}
    where \(x_0\) is constant to be determined in all cases. 
\end{remark}

\begin{eg}
    Consider the recurrence
    \[
        \begin{dcases}
            T(n) = 6T(n - 1) - 9T(n - 2) + 3^n\quad\text{for} n \geq 2\\
            T(0) = 0, T(1) = \frac{1}{2}.
        \end{dcases}
    \]
    Let \(T_0(n) = T_h(n) + T_p(n)\), where \(T_h(n) = 6T(n - 1) - 9T(n - 2)\). 
    
    For \(T_h(n)\), characteristic equation: \(x^2 = 6x - 9\), \(r_1 = 3\) with multiplicity \(m_1 = 2\). 
    
    Hence, we have \(T_h(n) = \theta_1 3^n + n\theta_2 3^n\), which is the homogeneous solution. 

    Since \(f(n) = 3^n\) and \(s = 3\) is also a root of the characteristic equation. Let \(T_p(n) = x_0 n^2 3^n\) (particular solution),
    \[
        \begin{aligned}
            x_0 n^2 3^n &= 6x_0 (n-1)^2 3^{n-1} - 9x_0 (n-2)^2 3^{n-2} + 3^n \\
            9x_0 n^2 &= 18x_0 (n-1)^2 - 9x_0 (n-2)^2 + 9\\
            9x_0 n^2 &= 18x_0 (n^2 - 2n + 1) - 9x_0 (n^2 - 4n + 4) + 9\\
            9x_0 n^2 &= 9x_0 n^2 - 18 x_0 + 9 \\
            x_0 &= \dfrac{1}{2}
        \end{aligned}
    \]
    Then, we have
    \[
        T_0(n) = T_h(n) + T_p(n) = \theta_1 3^n + n\theta_2 3^n + \dfrac{1}{2}n^2 3^n
    \]
    Using the initial conditions, we have
    \[
        \begin{aligned}
            0 &= T_0(0) = \theta_1 \\
            \dfrac{1}{2} &= T_0(1) = 3\theta_1 + 3\theta_2 + \dfrac{3}{2}
        \end{aligned}
    \]
    This gives \(\theta_1 = 0, \theta_2 = -\frac{1}{3}\). Hence, the solution to the recurrence is given by
    \[
        T(n) = -\dfrac{1}{3}n3^n + \dfrac{1}{2}n^2 3^n
    \]
\end{eg}

% END OF DOCUMENT
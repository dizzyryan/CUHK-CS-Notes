\chapter{Chebyshev's Inequality}

\section{Chebyshev's Inequality}
\begin{theorem}[Markov's Inequality]
    Suppose that random variable \(X \geq 0\) only takes non-negative values. Then, for every \(t > 0\) 
    \[
        \mathbb{P}(X \geq t) \leq \dfrac{\mathbb{E}[X]}{t}
    \]
    \begin{proof}
        \[
            \begin{aligned}
                \mathbb{E}[X] &= \mathbb{P}(X \geq t)\mathbb{E}[X \vert X \geq t] + \mathbb{P}(X < t)\mathbb{E}[X \vert X < t] \\
                \mathbb{E}[X] &\geq \mathbb{P}(X \geq t)\mathbb{E}[X \vert X \geq t] \\
                \mathbb{E}[X] &\geq t \times \mathbb{P}(X \geq t) \\
                \mathbb{P}(X \geq t) &\leq \dfrac{\mathbb{E}[X]}{t}
            \end{aligned}
        \]
    \end{proof}
\end{theorem}

Above could be understood as: if \(X \geq 0\) and \(\mathbb{E}[X]\) is small, then \(X\) is unlikely to be very large.

However, what if we take negative values for the random variable we use? 
\begin{theorem}[Chebyshev’s Inequality]
    Suppose that random variable \(X\) has a mean \(\mathbb{E}[X] = \mu\) and variance \(\Var[X] = \sigma^2\). Then, for every \(t > 0\), 
    \[
        \mathbb{P}(\vert X - \mu \vert \geq t) \leq \dfrac{\sigma^2}{t^2}
    \]
    \begin{proof}
        The proof follows from Markov's inequality. Let \(Z = (X - \mu)^2 \geq 0\) 
        \[
            \begin{aligned}
                \mathbb{P}(Z \geq \varepsilon^2) &\leq \dfrac{\mathbb{E}[Z]}{\varepsilon^2} \\
                &\leq \dfrac{\mathbb{E}[(X - \mu)^2]}{\varepsilon^2} \\
                &\leq \dfrac{\Var[X]}{\varepsilon^2} \\
                &\leq \dfrac{\sigma_x^2}{\varepsilon^2} \\
                \Longrightarrow \mathbb{P}(\vert X - \mu \vert \geq \varepsilon) &\leq \dfrac{\sigma^2}{\varepsilon^2}
            \end{aligned}
        \]
    \end{proof}
\end{theorem}

Again, by intuition, the above says that if the variance is small, then \(X\) is unlikely to be too far from the mean. 

\begin{eg}~ 

    (a) Find a bound on the probability of \(\leq 24\) heads in 64 coin flips?

    \textbf{Solution:} 

    Let \(X \sim\) Binomial(\(64, \frac{1}{2}\)) \(\rightarrow \mathbb{E}[X] = 64 \times \frac{1}{2} = 32\), \(\Var[X] = np(1 - p) = 64 \times \frac{1}{2} \times \frac{1}{2} = 16, \sigma_x = \sqrt{16} = 4 \) 
    \[
        \begin{aligned}
            \mathbb{P}(X \leq 24) &= \mathbb{P}(X - \mu \leq 24 - 32) \\
            &= \mathbb{P}(X - \mu \leq -2\sigma_x) \\
            &\leq \mathbb{P}(\vert X - \mu \vert \geq 2\sigma_x) \\
            &\leq \dfrac{\sigma_{x}^2}{(2\sigma_{x})^2} \\
            &= \dfrac{1}{4}
        \end{aligned}
    \]

    (b) Find a bound on the probability of \(\leq 24\) or \(\geq 40\) heads in 64 coin flips?

    \textbf{Solution:} 
    \[
        \begin{aligned}
            \mathbb{P}(\vert X - \mu \vert \geq 2\sigma_x) &= \mathbb{P}(\vert X - 32 \vert \geq 8) \\
            &= \mathbb{P}(X \leq 24 \text{ or } X \geq 40) \\
            &= \dfrac{\sigma_{x}^2}{(2\sigma_{x})^2} \\
            &= \dfrac{1}{4}
        \end{aligned}
    \]
\end{eg}

\section{Law of Large Numbers}
Imagine you observe Independent and Identically Distributed (IID)
random variables \(X_1, X_2, \cdots, X_n\) distributed according to PMF \(p\) with expected value \(\mathbb{E}[X_i] = \mu\). We have the sample mean \(M_n\), where 
\[
    M_n = \dfrac{X_1 + X_2 + \cdots + X_n}{n}.
\] 

Notice that \(M_n\) is a random variable since it is a function of random variables. It is, therefore, different from \(\mu\). 

The goal of the Law of Large Numbers is to show:
\[
    M_n = \dfrac{X_1 + X_2 + \cdots + X_n}{n} \longrightarrow \mu \quad (n \to \infty)
\]

\begin{proof}
    \[
        \mathbb{E}[\dfrac{X_1 + X_2 + \cdots + X_n}{n}] = \mathbb{E}[\dfrac{X_1}{n}] + \mathbb{E}[\dfrac{X_2}{n}] + \cdots + \mathbb{E}[\dfrac{X_n}{n}] = \dfrac{\mu}{n} + \dfrac{\mu}{n} + \cdots + \dfrac{\mu}{n} = \mu 
    \]
\end{proof}

For variance, we have 
\[
        \begin{aligned}
            \Var\left(\dfrac{X_1 + X_2 + \cdots + X_n}{n}\right) &= \Var\left(\dfrac{X_1}{n}\right) + \Var\left(\dfrac{X_2}{n}\right) + \cdots + \Var\left(\dfrac{X_n}{n}\right) \\
            &= \dfrac{\sigma^2}{n^2} + \dfrac{\sigma^2}{n^2} + \cdots + \dfrac{\sigma^2}{n^2} \\
            &= \dfrac{\sigma^2}{n}
        \end{aligned}
    \]


Suppose \(X_1, X_2, \cdots, X_n\) are IID with mean \(\mathbb{E}[X_i] = \mu\) and variance \(\Var[X_i] = \sigma^2\). Then, Chebyshev's inequality shows 
\[
        \mathbb{P}\left(\vert M_n - \mu \vert \geq \epsilon \right) \leq \dfrac{\Var[M_n]}{\epsilon^2} = \dfrac{\sigma^2}{n\epsilon^2}
\]

\newpage
\begin{theorem}[Weak Law of Large Numbers (WLLN)]
    Suppose \(X_1, X_2, \cdots\) is a sequence of IID random variables with expected value \(E[X_i] = \mu\) and finite variance\(\Var[X_i] < \infty\). Define \(\overline{X_n} = \frac{1}{n}(X_1 + \cdots + X_n)\). Then, for every \(\epsilon > 0\):
    \[
        \lim_{n \to \infty} \mathbb{P}(\vert \overline{X_n} - \mu \vert \geq \epsilon) = 0
    \]

    \begin{explanation}
        As the number of samples increases, i.e., \(n \rightarrow \infty\), the probability of the sample mean being greater than the actual mean will decrease and converge to 0 for a sufficiently large number of samples.
    \end{explanation}
\end{theorem}

\begin{eg}[The Pollster's Problem]
    Suppose in a referendum we want to determine the fraction of the population that will vote "yes," and we use \( p \) to describe this fraction. Now you take \( n \) samples of people to know their poll, and then you want to come up with an \( M_n \) that approximates the true \( p \) since you would not be able to know the actual \( p \). Then, what is the \( n \) that you should take to minimize the error?

    \textbf{Solution:} 
    Suppose the \(i\)-th randomly selected person polled \(X_i\), where  
    \[  
        X_i = \begin{dcases}  
            1, & \text{if yes} \\  
            0, & \text{if no}  
        \end{dcases} \quad,  
    \]  
    and we have the fraction of "yes" in our sample described as \(M_n\), where  
    \[  
        M_n = \dfrac{X_1 + X_2 + \cdots + X_n}{n}.  
    \]
    
    We would like a "small error," e.g., \(\vert M_n - p \vert < 0.01\). However, since we have no prior knowledge of \(p\), we can only ensure a small error by bounding it with a small probability, i.e., the probability of the error exceeding a certain value should not exceed a specified percentage. Then, we can try \(n = 10,000\). By Chebyshev's inequality, we have
    \[
        \mathbb{P}(\vert M_{10,000} - p \vert \geq 0.01) \leq \dfrac{\sigma^2}{n\epsilon^2} = \dfrac{p(1 - p)}{10^4 \cdot 10^{-4}} = p(1 - p)
    \] 
    Also, \(p\) remains unknown. Since it is uniformly distributed from 0 to 1, we see that \(p(1 - p)\) forms a parabola, with the maximum value at \(\frac{1}{4}\). Then we have
    \[
        \mathbb{P}(\vert M_{10,000} - p \vert \geq 0.01) \leq \dfrac{1}{4} = 25\%
    \]

    The above bound is, however, too large, and we want a smaller percentage, for example, \(5\%\). We can then determine the \(n\) we need by
    \[
        \dfrac{\frac{1}{4}}{n \cdot 10^{-4}} \leq \dfrac{5}{10^2} \Longleftrightarrow n \geq \dfrac{10^6}{20} = 50,000,
    \]
    which gives 
    \[
        \mathbb{P}(\vert M_{50,000} - p \vert \geq 0.01) \leq 0.05
    \]
\end{eg}

\section{Central Limit Theorem}
Suppose a random variable \(X\) has expected value \(\mathbb{E}[X] = \mu\) and variance \(\Var[X] = \sigma^2\). Then, random variable \(Z = \dfrac{1}{\sigma} (X - \mu)\) has zero expected value and unit variance: \(\mathbb{E}[Z] = 0\), \(\Var[Z] = 1\). 

Consider IID random variables \(X_1, X_2, \cdots, X_n\) has expected value \(\mathbb{E}[X_i] = \mu\) and variance \(\Var[X_i] = \sigma^2\) and define
\[
    S_n = X_1 + \cdots + X_n
\]
Then, random variable \(Z_n = \dfrac{S_n - n\mu}{\sigma\sqrt{n}}\) has zero expected value and unit variance. 

\begin{theorem}[Central Limit Theorem (CLT)]
    Consider IID random variables \(X_1, X_2, \cdots\) with expected value \(\mathbb{E}[X_i] = \mu\) and variance \(\Var[X_i] = \sigma^2\). Then, if we define
    \[
        Z_n = \dfrac{(X_1 + \cdots + X_n) - n\mu}{\sigma\sqrt{n}},
    \]
    we have 
    \[
        \mathbb{E}[Z_n] = \mathbb{E}\left[\dfrac{(X_1 + \cdots + X_n) - n\mu}{\sigma\sqrt{n}}\right] = \dfrac{n\mu - n\mu}{\sigma\sqrt{n}} = 0, \quad \Var[X_i] = \dfrac{\sigma\sqrt{n}}{\sigma\sqrt{n}} = 1.
    \]
    The CDF of random variable \(Z_n\) will converge to the CDF of a standard normal distribution \(\varPhi (z)\), i.e. for every \(z \in \mathbb{R}\)
    \[
        \lim_{n \to \infty} \mathbb{P}(Z_n \leq z) = \varPhi (z)
    \]
\end{theorem}
Therefore, according to the central limit theorem, the sum of many independent random numbers will approximately have a normal distribution.

\begin{eg}
    Suppose a package weighs \(X_i\), i.i.d. with \(X_i \sim \text{Exponential}\left(\frac{1}{2}\right)\). We now load the container with \(n = 100\) packages. Find 
    
    (1) \(\mathbb{P}(S_n \geq 210)\);
    
    (2) \(a\) such that \(\mathbb{P}(S_n \geq a) \approx 0.05\); 

    (3) largest \(n\) such that \(\mathbb{P}(S_n \geq 210) \approx 0.05\); 

    (4) \(\mathbb{P}(N \geq 100)\), where \(N\) is the number of packages loaded. We can load the container until weight exceed 210. 

    \textbf{Solution:} 

    (1)
    \[
        Z_n = \dfrac{S_n - n\mu}{\sigma\sqrt{n}}, \mu = \sigma = 2
    \]
    \[
        \mathbb{P}(S_n \geq 210) = \mathbb{P}(\dfrac{S_n - 200}{20} \geq \dfrac{210 - 200}{20}) = \mathbb{P}(Z \geq 0.5) = 1 - \mathbb{P}(Z < 0.5) = 1 - \varPhi (0.5) = 0.3085
    \]

    (2) 
    \[
        0.05 \approx \mathbb{P}(\dfrac{S_n - 200}{20} \geq \dfrac{a - 200}{20}) = \mathbb{P}(Z \geq \dfrac{a - 200}{20}) = 1 - \underbrace{\varPhi(\dfrac{a - 200}{20})}_{0.95}
    \]
    \[
        \dfrac{a - 200}{20} = 1.645 \Longrightarrow a = 232.9
    \]

    (3)
    \[
        \begin{aligned}
            \mathbb{P}(\dfrac{S_n - 2n}{2\sqrt{n}} \geq \dfrac{210 - 2n}{2\sqrt{n}}) &\approx 1 - \varPhi(\dfrac{210 - 2n}{2\sqrt{n}}) \approx 0.05 \\
            \dfrac{210 - 2n}{2\sqrt{n}} &= 1.645 \\
            n &= 89
        \end{aligned}
    \]

    (4)
    \[
        \mathbb{P}(N \geq 100) = \mathbb{P}\left(\sum_{i = 1}^{100} X_i \leq 210\right) \approx \varPhi\left(\dfrac{210 - 200}{20}\right) = \varPhi(0.5) = 0.6915
    \]

\end{eg}
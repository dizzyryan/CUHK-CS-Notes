\chapter{Binomial Coefficients}

\section{Introduction}
In this section, we introduce Binomial Coefficients.

\subsection{Combinations}
As it is introduced before,
\begin{definition} An r-combination of the n-element ground set $S_0$ is an \textbf{unordered selection} of $r$ elements from $S_0$. 
\[
	\binom{n}{r} = \dfrac{n!}{r!(n-r)!}
\]
\end{definition}

\subsection{Permutation}
Again, as it is introduced before, 
\begin{definition} An r-permutation of the n-element ground set $S_0$ is an \textbf{ordered selection} of $r$ elements from $S_0$. 
	\[
	P(n,r) = \dfrac{n!}{(n-r)!}
	\]
\end{definition}

\subsection{Binomial Identities}
\begin{proposition} For any integers \(m, r \geq 0\) with \(0 \leq r \leq n\),
	\[
		\binom{n}{r} = \binom{n}{n - r}
	\]
\end{proposition}

We have two ways to prove this proposition, namely Algebraic Proof and Combinatorial Proof.

\begin{proof}[Algebraic Proof]
	\[
		\binom{n}{n-r} = \dfrac{n!}{(n-r)!(n-n+r)!} = \dfrac{n!}{r!(n-r)!} = \binom{n}{r}
	\]
\end{proof}

\begin{proof}[Combinatorial Proof]
	Both side of the identity are supposed to be two different ways of solving a counting problem. We can define the counting problem as counting the number of different unordered selections of \(r\) elements from an \(n\)-element ground set. Then, we can define the RHS as the selecting number to be excluded. Then this identity holds.  
\end{proof}

\subsection{Pascal's Identity}
\begin{theorem}
	For any integers \(n,\ r \geq 0\) with \(1 \leq r \leq n-1\),
	\[
		\binom{n}{r} = \binom{n-1}{r} + \binom{n-1}{r-1} \mathbf{\text{ OR }} \binom{n+1}{r} = \binom{n}{r} + \binom{n}{r-1}
	\]  
\end{theorem}

Again, we can prove this theorem by two ways.

\begin{proof}[Algebraic Proof]
	\[
		\begin{aligned}
			RHS &= \dfrac{(n-1)!}{r!(n-r-1)!} + \dfrac{(n-1)!}{(r-1)!(n-1-r+1)!} \\
			&= \dfrac{(n-1)!}{r!(n-r-1)!} + \dfrac{(n-1)!}{(r-1)!(n-r)!} \\
			&= \dfrac{(n-1)!}{(r-1)!(n-r-1)!} \left(\dfrac{1}{r} + \dfrac{1}{n-r}\right) \\
			&= \dfrac{(n-1)!}{(r-1)!(n-r-1)!} \left(\dfrac{n}{r(n-r)}\right) \\
			&= \dfrac{n!}{r!(n-r)!} \\
			&= \binom{n}{r}
		\end{aligned}
	\]
\end{proof}

\begin{proof}[Combinatorial Proof] 
	Recall that \(\binom{n}{k}\) equals the number of subsets with \(k\) elements from a set with \(n\) elements. Suppose one particular element is uniquely labeled \(X\) in a set with \(n\) elements.
	
	To construct a subset of \(k\) elements containing \(X\), include \(X\) and choose \(k-1\) elements from the remaining \(n-1\) elements in the set. There are \(\binom{n-1}{k-1}\) such subsets.

	To construct a subset of \(k\) elements \textbf{not} containing \(X\), choose \(k\) elements from the remaining \(n-1\) elements in the set. There are \(\binom{n-1}{k}\) such subsets.

	Every subset of \(k\) elements either contains \(X\) or not. The total number of subsets with \(k\) elements in a set of \(n\) elements is the sum of the number of subsets containing \(X\) and the number of subsets that do not contain \(X\), \(\binom{n-1}{k-1} + \binom{n-1}{k}\).

	This equals \(\binom{n}{k}\).
\end{proof}

\chapter{Dynamic Programming}
\section{Pitfall of Recursion}
Let \(A\) be an array of \(n\) positive integers, with function
\[
  f(k) = \begin{dcases}
   0 , &\text{ if } k = 0 \\
   \max_{i = 1}^k (A[i] + f(k - i)), &\text{ if } 1 \leq k \leq n 
  \end{dcases}
\]
To compute \(f(n)\), we can utilize the following:

\begin{algorithm}[H]
	\DontPrintSemicolon{}
	\caption{\(f(n)\)}
	\BlankLine
  \If{\(k = 0\)}
  {
    \Return{0}\;
  }
  \(\textit{ans} \gets -\infty\)\;
	\For{\(i \gets 1\) \KwTo \(k\)}{
		\(\textit{tmp} \gets A[i] + f(k - i)\)\;
    \If{\(\textit{tmp} > \textit{ans}\)}
    {
      \(\textit{ans} \gets \textit{tmp}\)\;
    }
	}
	\Return{\(\textit{ans}\)}\;
\end{algorithm}

This recursion gives \(\Omega(2^n)\) because it computes \(f(x)\) for the same \(x\) repeatedly. A recursive algorithm does considerable redundant work if the same sub-problem is encountered over and over again.

We resolve such by dynamic programming, where we resolve sub-problems according to a certain order, remembering the output of every sub-problem to avoid re-computation.

For the same problem, we consider solving the problem with order \(f(1), f(2), \cdots, f(n)\). Then we need \(\mathcal{O} (n^2)\) time in total. 

\begin{algorithm}[H]
	\DontPrintSemicolon{}
	\caption{Dynamic programming for \(f(n)\)}
  \label{DynRodCutting}
	\BlankLine
  Initialize: \(\textit{ans}[n], \textit{ans}[0] \gets 0\)\; 
  \For{\(k \gets 1\) \KwTo \(n\)}{
    \(\textit{ans}[k] \gets -\infty\)\;
    \For{\(i \gets 1\) \KwTo \(k\)}{
      \(\textit{tmp} \gets A[i] + \textit{ans}[k - i]\)\;
      \If{\(\textit{tmp} > \textit{ans}[k]\)}{
        \(\textit{ans}[k] \gets \textit{tmp}\)\;
      }
    }
  }
\end{algorithm}

In the following, we will see examples of dynamic programming, where the core idea is to break the problem into sub-problems that can be solved in a \textbf{nice order}, so that later sub-problems can reuse the solutions of earlier ones.

The key to solving a problem is identifying its recursive structure — how the original problem relates to its sub-problems — which then guides the design of a dynamic programming algorithm.

\section{Basic Techniques}
In algorithm, there are three basic techniques we can utilize. 

\subsection{Recursion}
When dealing with a sub-problem (the same problem but with a smaller input), consider it solved, and use the sub-problem’s output to continue the algorithm design.

\begin{eg}[The Hanoi Tower Problem]
  There are 3 rods A, B, and C. On rod A, \(n\) disks of different sizes are stacked, such that no disk of a larger size is above a disk of a smaller size. The other two rods are empty. 

  It is allowed to move the top-most disk of a rod to another, while no disk of a larger size can be put above a disk of a smaller size. The goal is to design an algorithm to move all the disks to rod B.

  To move the largest disk, i.e., the \(n\)-th disk, to rod B, we first need to move all disks above it to rod C. For all disks, we use this method; thus, if we ignore the last disk, the remaining problem is to move \(n - 1\) disks. This is the same problem with a smaller size. 

  Suppose that the algorithm performs \(f(n)\) operations to solve a problem of size \(n\), where \(f(1) = 1\). By recursion, we have 
  \[
    f(n) \geq 1 + 2 \times f(n - 1).
  \]
  \href{https://www.ryanc.wtf/files/ENGG2440.pdf#page=14}{Solving} this gives 
  \[
    f(n) \geq 2^n - 1.
  \]
  Thus, the best time complexity for solving this problem is \(\Omega(2^n)\), where \(n\) is the number of disks. 
\end{eg}

\subsection{Repeating till Success}

Given a set \(S\) of \(n\) integers in an array and an integer \(k \in [1, n]\), we want to find the \(k\)-th smallest integer of \(S\). 

\begin{definition}[Rank]
    The rank of an integer \(v \in S\) is the number of elements in \(S\) smaller than or equal to \(v\). 
\end{definition}

For example, suppose that \(S = (53, 92, 85, 23, 35, 12, 68, 74)\), then the rank of 53 is 4. 

We can obtain the rank of \(v\) in \(\mathcal{O}(|S|)\) time. 

\begin{eg}[Sub-problem of The \(k\)-Selection Problem]
  Assume \(n\) is a multiple of 3. We want to obtain a subproblem of size at most \(\frac{2n}{3}\) with exactly the same result as the original problem. 

  Our goal is to produce a set \(S^{\prime}\) and an integer \(k^{\prime}\) such that \(|S^{\prime}| \leq \frac{2n}{3}\), \(k^{\prime} \in [1, |S^{\prime}|]\), and the element with rank \(k^{\prime}\) in \(S^{\prime}\) is the element with rank \(k\) in \(S\), where it is possible that \(k \neq k^{\prime}\). 

  Consider the following algorithm \(A_{sub}\): 

  We first take an element \(v \in S\) uniformly at random. Then we divide \(S\) into \(S_1\) and \(S_2\), where \(S_1\) is the set of elements in \(S\) that are less than or equal to \(v\), and \(S_2\) is the set of elements in \(S\) greater than \(v\). 

  If \(|S_1| \geq k\), then return \(S^{\prime} = S_1\) and \(k^{\prime} = k\). Otherwise, return \(S^{\prime} = S_2\) and \(k^{\prime} = k - |S_1|\). 

  This algorithm succeeds if \(|S^{\prime}| \leq \frac{2n}{3}\), and fails otherwise. 

  We repeat this algorithm until it succeeds. 

  \begin{lemma}
    The algorithm succeeds with probability at least \(\frac{1}{3}\). 
  \end{lemma}

  \begin{proof}
    Let \(S\) be a set of size \(n\) and let \(v \in S\) be chosen uniformly at random. Let the rank of \(v\) in \(S\) be \(\text{rank}(v)\). 

    Define \(S_1 = \{x \in S : x \leq v\}\), \(S_2 = \{x \in S : x > v\}\). If \(|S_1| \geq k\), return \(S^{\prime} = S_1\) and \(k^{\prime} = k\); otherwise \(S^{\prime} = S_2\) and \(k^{\prime} = k - |S_1|\). The sub-problem succeeds if \(|S^{\prime}| \leq \frac{2n}{3}\). 

    If \(|S_1| \geq k\), then \(|S^{\prime}| = |S_1| = \text{rank}(v)\), and success requires
    \[
      \text{rank}(v) \leq \dfrac{2n}{3}. 
    \]
    If \(|S_1| < k\), then \(|S^{\prime}| = |S_2| = n - \text{rank}(v)\), and success requires
    \[
      n - \text{rank}(v) \leq \dfrac{2n}{3} \Longrightarrow \text{rank}(v) \geq \dfrac{n}{3}.
    \]
    Then we have 
    \[
      \text{rank}(v) \in \left[\dfrac{n}{3}, \dfrac{2n}{3}\right].
    \]
    Since \(v\) is chosen uniformly at random, every rank is equally likely, and the number of ranks in the success range is at least \(\frac{2n}{3} - \frac{n}{3} = \frac{n}{3}\). Thus we have 
    \[
      \mathbb{P}[\text{success}] \geq \dfrac{\frac{n}{3}}{n} = \dfrac{1}{3}.
    \]
  \end{proof}

  \begin{remark}
    Choosing \(\frac{2n}{3}\) is not mandatory, but rather a convenient threshold since it gives a smaller sub-problem. For example, if we say it succeeds when \(|S^{\prime}| \leq \frac{9n}{10}\), we have the range \([\frac{n}{10}, \frac{9n}{10}]\), giving the probability of success \(\frac{8}{10}\), which is high. However, this is simply because we have a larger range to choose from, making the sub-problem larger. 
  \end{remark}
\end{eg}

In general, if an algorithm succeeds with a probability at least \(c > 0\), then the number of repeats needed for the algorithm to succeed for the first time is at most \(\frac{1}{c}\) in expectation. 

\subsection{Geometric Series}
A geometric sequence is an infinite sequence of the form 
\[
  n, cn, c^2 n, c^3 n, \dots 
\]
where \(n\) is a positive number and \(c\) is a constant satisfying \(0 < c < 1\). It holds in general that 
\[
  \sum_{i = 0}^{\infty} c^i n = \lim_{i \to \infty} n \cdot \dfrac{1 - c^i}{1 - c} = \dfrac{n}{1 - c} = \mathcal{O}(n).
\]
The summation \(\sum_{i = 0}^{\infty} c^i n\) is called a geometric series. 

Consider again the \(k\)-th selection problem. Using the previous repeating technique, we can convert the problem into a subproblem with size at most \(\lceil \frac{2n}{3} \rceil\) in \(\mathcal{O}(n)\) expected time. We can use the geometric series to find the expected running time: 
\[
  \begin{aligned}
    a \cdot n + a \cdot \frac{2}{3} \cdot n + a \cdot \left(\frac{2}{3}\right)^2 \cdot n + \cdots &= a \cdot n + a \cdot \sum_{i = 1}^{\infty} \left(\frac{2}{3}\right)^i \cdot n \\
    &= a \cdot n + a \cdot \mathcal{O}(n) \\
    &= \mathcal{O}(n).
  \end{aligned}
\]

Next, we analyze the running time of \(A_{\text{sub}}\). It takes \(\mathcal{O}(1)\) to select the element \(v\), and dividing \(S\) into \(S_1\) and \(S_2\) takes \(\mathcal{O}(n)\). Comparing \(|S_1|, |S_2|\), and \(k\) also takes \(\mathcal{O}(n)\). Thus, by the linearity of expectation, every conversion takes \(\mathcal{O}(n)\) running time in expectation. 

By the previous lemma, we need to repeat the algorithm 3 times in expectation until it succeeds, and each execution takes \(\mathcal{O}(n)\) time. Therefore, the \(k\)-th selection algorithm takes \(\mathcal{O}(n)\) time in expectation.  

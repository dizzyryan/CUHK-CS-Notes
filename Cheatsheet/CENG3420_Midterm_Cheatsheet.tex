% ---------------------- setting ---------------------- %
\documentclass[10pt,landscape,a4paper]{article}
\input{../Math.tex}
\usepackage[margin=0.5in]{geometry}
\usepackage{multicol}
\usepackage{fancyhdr}
\usepackage{amsmath, amsfonts, mathtools, amsthm, amssymb, mathrsfs}
\usepackage{lipsum}
\usepackage{blindtext}
\linespread{1.2}

\setlength\parindent{0pt}

\setlength{\columnseprule}{0.5pt}
\pagestyle{fancy}
\renewcommand{\headrulewidth}{0pt}
\fancyhf{}
\pagestyle{empty}

\makeatletter
\renewcommand{\section}{\@startsection{section}{1}{0mm}%
                                {-1ex plus -.5ex minus -.2ex}%
                                {0.5ex plus .2ex}%x
                                {\normalfont\large\bfseries}}
\renewcommand{\subsection}{\@startsection{subsection}{2}{0mm}%
                                {-1explus -.5ex minus -.2ex}%
                                {0.5ex plus .2ex}%
                                {\normalfont\normalsize\bfseries}}
\renewcommand{\subsubsection}{\@startsection{subsubsection}{3}{0mm}%
                                {-1ex plus -.5ex minus -.2ex}%
                                {1ex plus .2ex}%
                                {\normalfont\small\bfseries}}
\makeatother

\usepackage{etoolbox}
\makeatletter
\preto{\@verbatim}{\topsep=0pt \partopsep=0pt }
\makeatother

\begin{document}
\begin{multicols}{3}
% ---------------------- Content ---------------------- %
\textbf{CPU design}

1. Yield = The proportion of working dies per wafer. 

2. Cost per die \(= \dfrac{\text{Cost per wafer}}{\text{Dies per wafer }\times\text{ Yield}}\) 

3. Dies per wafer \(\approx \dfrac{\text{Wafer area}}{\text{Die area}}\) (since wafers are circle)

4. Yield = \(\dfrac{1}{\left[1 + (\frac{\text{Defects per area} \times \text{Die area}}{2})\right]^2}\) 

\(= \text{Defects per unit area} \times \text{Die area}\)

\(\text{Power } = \text{ Capacitive load } \times \text{ Voltage}^2 \times \text{ Frequency}\) 

\textbf{ISA}

Components: processor, I/O, mem, and network.

ISA: formal specification of the instruction set that is implemented in the machine hardware. 

1. Simplicity favours regularity; 2. Smaller is faster; 3. Make the common case fast; 4. Good design demands good compromises; 

Important registers: 

(ra: usually PC + 4); (sp: must be aligned to 4 bytes); (gp: holds the base address of global variables)

\textbf{Arithmetic}

rs1 and rs2 fields kept in the same place: imm field in S-type separated
\begin{verbatim}
destination = source1 op source2
\end{verbatim}
In I format, values range: \(-2^{11}\) to \(2^{11} - 1\). 

Load 32 bits:
\begin{verbatim}
lui t0, 1010 1010 1010 1010 1010b
ori t0, t0, 1010 1010 1010b
\end{verbatim}

logical shift: fill the vacancy with zeros
\begin{verbatim}
slli t2, s0, 8   # t2 = s0 << 8 bits
srli t2, s0, 8   # t2 = s0 >> 8 bits
\end{verbatim}
\begin{verbatim}
lw t0, 4(s3)   # load word from mem to reg
sw t0, 8(s3)   # store word from reg to mem
(loaded or stored using a 5-bit address)
\end{verbatim}

\textcolor{red}{NOTE: Address is byte-base: increment 4 when accessing reg}

Little Endian: rightmost byte is the most significant byte.

\verb|lb| places the byte from mem into the rightmost 8 bits of the dest reg and signed extension.
\begin{verbatim}
lb t0, 1(s3)   # load byte from memory
sb t0, 6(s3)   # store byte to memory
\end{verbatim}

stack grows from high address to low address

2's complement: complement all the bits and then add 1

\(6 = 00...\ 0110_2 \Rightarrow 11...\ 1001_2 + 1 \Rightarrow 11 ...\ 1010 = -6\)

\(n\)-bit signed binary: \([2^{n-1} - 1, -2^{n-1}]\)

\textbf{ALU}

32-bit signed numbers: range from \(2^{31} - 1\) to \(-2^{31}\)

If the bit string represents address: 0 to \(2^{32} - 1\).

Sign extension copies the most significant bit into the other bits to preserve the sign of the number.

Ripple Carry Adder: connect all adders in sequence, slow because each bit's carry-out depends on the previous bit's carry-in, leading to a cumulative delay.

Overflow: adding two positive numbers yields a negative / adding two negative numbers gives a positive / subtracting a negative from a positive gives a negative / subtracting a positive from a negative gives a positive.

\verb|mul|: 32-bit \(\times\) 32-bit multiplication and places the lower 32 bits in the destination register. \verb|mulh|, \verb|mulhu|, and \verb|mulhsu| perform the same multiplication but return the upper 32 bits of the full 64-bit product.

Logical shifts fill with zeros, while arithmetic right shifts fill with the sign bit.

\textbf{Floating}
\(
  \underbrace{6.6254}_{\text{Mantissa (always +)}} \times {\underbrace{10}_{\text{Base}}}^{-27} \Longleftarrow \pm 1.M \times 2^{E^{\prime} - 127}
\)
Structure: S | E' | M

S: Sign bit; E': 8-bit signed exponent; M: mantissa 

e.g. \(\verb|40C0000|_{\verb|16|}\) in decimal

1. 40C0000 = \verb|0 10000001 10000000000000000000000|
  
2. Sign bit (0): Positive (\(+\))

3. Exponent: \(10000001_2 - 127 = 129 - 127 = 2\)

4. Mantissa: \(1.1000000000... = 1 + 1 \times 2^{-1} = 1.5\)

Result: \(1.5 \times 2^2 =6\)

e.g. \(-0.5_{10}\) in binary 

1. Sign bit: 1 
  
2. Mantissa: \(0.5 = 1.0 \times 2^{-1}\)  
  
3. Exponent: \(127 - 1 = 126 = 0111 1110\)

Result: \(-0.5_{10} = \verb|10111111000000000000000000000000|\)

\(E = 0, M = 0\): 0; \(E = 0, M \neq 0\): denormalized number, which is \(\pm 0.M \times 2^{-126}\); \(E = 1\ldots 1\), \(M = 0\): \(\pm \infty\), depending on the sign; \(E = 1\ldots 1\), \(M \neq 0\): \verb|NaN| (Not a Number).

\textbf{Datapath}

combinational: ALU; state: memory

The partition of \verb|imm| field: align the \verb|imm| with other instruction types - more efficient implementation of control units.

The instruction is decoded in the path between the Instruction Memory and Register File.

\textbf{Pipeline}

clock cycle: timed to accommodate the slowest instruction: instr take same amount of time

\(\text{CPU time} = \text{CPI} \times \text{CC} \times \text{IC}\), CPI = cycles per instruction, CC = clock cycle time, IC = instruction count.

IF: Instruction fetch and PC update

ID: Instruction decode and register file read

EXE: Execution or address calculation

MEM: Data memory access (only in loads and stores)

WB: Write the result data back into the register file

instruction latency (time from start of instr to completion) is not reduced.

State registers - between pipeline stage. Register: flip-flop, data moves in at rising edge 

Structural hazards: conflicts in the use of a resource

- separating instruction and data memories

- reads in the 2nd half of the cycle and writes in the 1st half
% ---------------------- END ---------------------- %
\vfill\break
\end{multicols}
\end{document}
\documentclass[12pt]{article}
\usepackage[margin=1in]{geometry}
\usepackage[all]{xy}
\usepackage{amsmath,amsthm,amssymb,color,latexsym}
\usepackage{geometry}        
\geometry{a4paper}  
\usepackage{systeme, afterpage}
\usepackage{graphicx}
\usepackage{gauss}
\usepackage{xparse}
\usepackage{anyfontsize}

\setlength\parindent{0pt}
\linespread{1.25}

\begin{document}

\section*{Q1}
1. Given that a die has 0.2 defects on average, Defects per area \(\times\) Die area = defects = 0.2 
\[
  \text{Yield1} = \dfrac{1}{\left[1 + \left(\frac{\text{Defects per area \(\times\) Die area}}{2}\right)\right]^2} = \dfrac{1}{\left(1 + \frac{0.2}{2}\right)^2}= 0.826
\] 

2. If the average defects of a die is reduced to 0.1, 
\[
  \text{Yield} = \dfrac{1}{\left(1 + \frac{0.1}{2}\right)^2} = 0.9070
\]
\[
    \text{Cost per die} = \dfrac{\text{Cost per wafer}}{\text{Dies per wafer} \times \text{Yield1}} = \dfrac{\text{Cost per wafer}}{\text{Dies per wafer} \times 0.826}
\]
\[
  \text{New cost per die} = \dfrac{\text{Cost per wafer}}{\text{Dies per wafer} \times \text{Yield}} = \dfrac{\text{Cost per wafer}}{\text{Dies per wafer} \times 0.9070}
\]
\[
  \text{Ratio} = \dfrac{\text{New cost per die} - \text{Cost per die}}{\text{Cost per die}} = \dfrac{\text{New cost per die}}{\text{Cost per die}} - 1 = 0.09753
\]
\[
  \text{Money saved} = 10,000,000 \times 0.09753 = 975,300 \text{HKD}
\]
Therefore, HKD\$975300 could be saved.

\section*{Q2}
\[
  \dfrac{\text{Power}_\text{new}}{\text{Power}_\text{old}} = \dfrac{75\% \times (80\%)^2 \times (90\%)}{1} = 0.432
\]

Since the ratio is less than 1, it indicates that less power will be required to perform the same task.

\section*{Q3}
1. 
\begin{verbatim}
  lw a1, 12(a0)
\end{verbatim}

2. 
\begin{verbatim}
  addi a2, a1, -32
\end{verbatim}

3. 
For \(\verb|t2|/32\), we can use 
\begin{verbatim}
  srli t3, t2, 5
\end{verbatim}

\quad For \(\verb|t2|\%32\), we can use 
\begin{verbatim}
  andi t4, t2, 0x1F
\end{verbatim}

\section*{Q4}
1. 
\begin{verbatim}
  slli t3, t1, 8 # t3 = 0xABCDEF00
  srli t3, t3, 8 # t3 = 0x00ABCDEF
\end{verbatim}
\quad Value of \verb|t3 = 0x00ABCDEF|

2. 
\begin{verbatim}
  slli t3, t2, 8 # t3 = 0xA0000000
  srai t3, t3, 8 # t3 = 0xFFA00000
\end{verbatim}
\quad Value of \verb|t3 = 0xFFA00000|

3. 
\begin{verbatim}
  xor t3, t1, t2
  sw t3, 0(a0)
  sb t3, 8(a0)
\end{verbatim}

\section*{Q5}
1. 4 times

2. \verb|4 12 24 24|

3. It finds the factorial of \verb|t1|

\section*{Q6}
\begin{verbatim}
  addi sp, sp, -16 # assign 4 words of data
  ... 
  sw ra, 12(sp) # save return address in ra
  sw a0, 8(sp) # save a0 to stack
  sw a1, 4(sp) # save a1 to stack
  ... 
  lw a1, 4(sp) # restore a1 from stack
  lw a0, 8(sp) # restore a0 from stack
  lw ra, 12(sp) # restore ra from stack
  ...
  addi sp, sp, 16 # free the stack
\end{verbatim}

\vfill\centering\textbf{- END -}
\end{document}
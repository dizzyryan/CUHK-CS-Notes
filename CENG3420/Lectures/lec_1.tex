\chapter{Introduction}

This course is about how computers work. 

\section{The Manufacturing Process of Integrated Circuit}

For this chapter, only a few calculations need to be considered:

1. Yield = The proportion of working dies per wafer. 

2. Cost per die \(= \dfrac{\text{Cost per wafer}}{\text{Dies per wafer }\times\text{ Yield}}\) 

3. Dies per wafer \(\approx \dfrac{\text{Wafer area}}{\text{Die area}}\) (since wafers are circle)

4. Yield = \(\dfrac{1}{\left[1 + (\frac{\text{Defects per area} \times \text{Die area}}{2})\right]^2}\) 

\begin{remark}
  Note that the defects on average \(= \text{Defects per unit area} \times \text{Die area}\).
\end{remark}

\section{Power}
\[
  \text{Power } = \text{ Capacitive load } \times \text{ Voltage}^2 \times \text{ Frequency}
\]

\begin{eg}
  For a simple processor, the capacitive load is reduced by 15\%, voltage is reduced by 15\%, and the frequency remains the same. Then, how much power consumption can be reduced? 

  \textbf{Solution:} 
  \[
    1 - (1 - 15\%) \times (1 - 15\%) \times 1 = 27.75\%
  \]
  Thus, 27.75\% of the power consumption can be reduced. 
\end{eg}
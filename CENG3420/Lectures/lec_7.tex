\chapter{Floating Numbers}
We discussed the representation of integers in previous chapters, and the representation of floating-point numbers is more complex.

However, we can break a floating-point number into parts. For example, consider 
\[
  \underbrace{6.6254}_{\text{Mantissa (always positive)}} \times {\underbrace{10}_{\text{Base}}}^{\overbrace{-27}^{\text{Exponent}}}
\]

We have:

1. Mantissa: A normalized number with a certain level of accuracy (e.g., 6.6254).

2. Exponent: A scale factor that determines the position of the decimal point (e.g., \(10^{-27}\)).

3. Sign bit: Indicates whether the number is positive or negative.

We normalize the mantissa to fall within the range \([1, R)\), where \(R\) is the base. For instance, in the case of a binary base, this range would be \([1, 2)\).

In IEEE Standard 754 Single Precision, we have 
\begin{center}
\begin{bytefield}[leftcurly=., leftcurlyspace=0pt, bitwidth=12pt]{32}
  \bitbox{1}{\(S\)} & \bitbox{8}{\(E^{\prime}\)} & \bitbox{23}{\(M\)}
\end{bytefield}
\end{center}

Here, \(S\) represents the sign bit, where 0 indicates a positive number and 1 indicates a negative number. \(E^{\prime}\) is the 8-bit signed exponent, represented in excess-127 notation, ranging from \(-127\) to \(128\). \(M\) is the 23-bit mantissa fraction. The value is thus represented as \(\pm 1.M \times 2^{E^{\prime} - 127}\).
\begin{remark}
  Minimum exponent \(= 1 - 127 = -126\); Maximum exponent \(= 254 - 127 = 127\)
\end{remark}

For double precision, we use 64 bits. \(E^{\prime}\) is the 11-bit signed exponent, represented in excess-1023 notation, and \(M\) is the 52-bit mantissa fraction.

\begin{eg}
  What is the IEEE single precision number \(\verb|40C0000|_{\verb|16|}\) in decimal?

  \textbf{Solution:} 
  First, convert the hexadecimal number \(\verb|40C0000|_{\verb|16|}\) to binary:  
  \[
  \verb|40C0000|_{\verb|16|} = \verb|0100 0000 1100 0000 0000 0000 0000 0000|_{\verb|2|}
  \]
  
  Sign bit (0): Positive (\(+\))

  Exponent: \(\verb|10000001|_{\verb|2|} - 127 = 129 - 127 = 2\)

  Mantissa: \(\verb|1.100 0000 0000 0000 0000 0000|_{\verb|2|} = 1 + 1 \times 2^{-1} = 1.5\)

  Therefore, the result is:  
  \[
  1.5 \times 2^2 = 6_{10}
  \]
\end{eg}

\begin{eg}
  What is \(-0.5_{10}\) in IEEE single precision binary floating-point format? 

  \textbf{Solution:}  
  Sign bit: 1 (since the number is negative)  
  
  Mantissa: \(0.5_{10}  = 1.0 \times 2^{-1} = 0.1_{2c} \)  
  
  Exponent: \(127 - 1 = 126 = 0111 1110\)

  Thus, in binary:
  \[
    -0.5_{10} = \verb|1011 1111 0000 0000 0000 0000 0000 0000|_{\verb|2|}
  \]
\end{eg}

\begin{remark}
  Exponents with all 0's or all 1's have special meanings in IEEE floating-point representation:
  
  - \(E = 0, M = 0\): Represents 0.

  - \(E = 0, M \neq 0\): Represents a denormalized number, which is \(\pm 0.M \times 2^{-126}\).

  - \(E = 1\ldots 1\), \(M = 0\): Represents \(\pm \infty\), depending on the sign.

  - \(E = 1\ldots 1\), \(M \neq 0\): Represents \verb|NaN| (Not a Number).
\end{remark}
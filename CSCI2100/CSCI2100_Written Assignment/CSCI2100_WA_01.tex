\documentclass[12pt]{article}
\usepackage[margin=1in]{geometry}
\usepackage[all]{xy}
\usepackage{amsmath,amsthm,amssymb,color,latexsym}
\usepackage{geometry}        
\geometry{a4paper}  
\usepackage{systeme, afterpage}
\usepackage{graphicx}
\usepackage{gauss}
\usepackage{xparse}
\usepackage{anyfontsize}

\setlength\parindent{0pt}
\linespread{1.25}

\begin{document}

\section*{Problem 1.1}

\subsection*{(1)}
\[
  \sum_{i = 0} ^{\infty} \dfrac{1}{4^i} = \sum_{i = 0} ^{\infty} (\dfrac{1}{4})^i = \dfrac{(\frac{1}{4})^{n+1} - 1}{\frac{1}{4} - 1} = \dfrac{4 - 4^{-\infty}}{3} = \dfrac{4}{3}
\]

\subsection*{(9)}
\[
  \sum_{i = 0}^n i^3 = \dfrac{n^{2} (n^{2} + 2 n + 1)}{4}
\]

\subsection*{(13)}
Assume that \(2^{2n} = O(2^n)\), then we have \(2^{2n} \leq c \cdot 2^n\) for \(c > 0\) and an \(n_0 \geq 0\) for all \(n \geq n_0\). 
\[
  \begin{aligned}
    2^{2n} &\leq c \cdot 2^n \\
    2^{n} &\leq c
  \end{aligned}
\]
Since for \(n_0 \geq 0\), there does not exist a fixed \(c\) such that \(2^n \leq c\), by contradiction, it can be proved that \(2^{2n} \neq O(2^n)\).

\section*{Problem 1.3}

\subsection*{(3)}
Let \(T_0(n) = T_h(n) + T_p(n)\), where \(T_h(n) = aT(n - 1)\). 

For \(T_h(n)\), characteristic equation \(x = a\), then we have \(T_h(n) = \theta a^n\). 

Since \(f(n) = bn^c, s = n\), let \(T_p(n) = x_0 n^c \). 
\[
  \begin{aligned}
    x_0 n^c &= ax_0 (n - 1)^c + bn^c \\
    x_0 &= \dfrac{bn^c}{n^c - a(n - 1)^c}
  \end{aligned}
\]

Then, we have 
\[
  T_0(n) = \theta a^n + \dfrac{bn^{2c}}{n^c - a(n - 1)^c}
\]

Since \(T(1) = 1\), we have 
\[
  1 = \theta \cdot a + b \Longrightarrow \theta = \dfrac{1 - b}{a}
\]

Hence, the solution to the recurrence is given by 
\[
  T(n) = (1 - b)a^{n-1}  + \dfrac{bn^{2c}}{n^c - a(n - 1)^c}
\]

\subsection*{(8)}
Let \(T_0(n) = T_h(n) + T_p(n)\), where \(T_h(n) = 3T(n - 1)\). 

For \(T_h(n)\), characteristic equation \(x = 3\), then we have \(T_h(n) = \theta 3^n\). 

Since \(f(n) = 2\), let \(T_p(n) = x\). 
\[
  \begin{aligned}
    x &= 3x + 2 \\
    x &= -1
  \end{aligned}
\]

Then, we have 
\[
  T_0(n) = \theta 3^n - 1
\]

Since \(T(1) = 1\), we have 
\[
  1 = 3\theta - 1
\]

This gives \(\theta = \frac{2}{3}\). Hence, the solution to the recurrence is given by 
\[
  T(n) = 2 \cdot 3^{n-1}  - 1
\]

\section*{Problem 1.4}

\subsection*{(2)}
Let \(P(n)\) be the predicate 
\[
  P(n) : \sum_{i = 0}^n 2^i = 2^{n+1} - 1
\]

For \(n = 0\), \(L.H.S = 2^0 = 1, R.H.S = 2^{0+1} - 1 = 1\), which holds true. 

Assume that 
\[
\sum_{i = 0}^n 2^i = 2^{n+1} - 1. 
\]

For \(P(n + 1)\), 
\[
  \begin{aligned}
    \sum_{i = 0}^{n+1}  2^i &= \sum_{i = 0}^n 2^i + 2^{n+1} \\
    &= 2^{n+1} - 1 + 2^{n+1} \\
    &= 2 \times 2^{n+1} - 1\\
    &= 2^{(n+1)+1} - 1\\
    &= R.H.S
  \end{aligned}
\]
which shows that \(P(n + 1)\) is also true. 
Hence, by the principle of MI, we can conclude that \(P(n)\) is true for all integers \(n \geq 1\).

\subsection*{(4)}
Let \(f(x) = (n + 1)^2, g(x) = n^2\).
\[
  f(x) = n^2 + 2n + 1 \leq n^2 + 2n^2 + n^2 = 4n^{2}
\]

Thus, by taking \(c = 4, n_0 = 1\), for \(n > n_0\), \((n + 1)^2 = O(n^2)\). 

\vfill\centering\textbf{- END -}
\end{document}

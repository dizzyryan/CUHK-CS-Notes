\chapter{Introduction}

\section{Theory of Computation}
In the theory of computation, we study the following:

1. \textbf{Formal Languages}

This is the abstraction of the general characteristics of programming languages. It consists of a set of symbols and some rules, i.e. strings and grammar of formation, and these symbols are then combined into sentences.

2. \textbf{Automata Theory}

This is the study of the dynamic behaviours of ``discrete-parameter information systems'' in the form of ``abstract computing devices.''

Types of automata are distinguished by their temporary memory.

For finite automata, there is no temporary memory. Pushdown automata use a stack, and Turing machines use random-access memory. The computational power of finite automata is the smallest, while Turing machines have the highest, with pushdown automata in between.

There are three major models of automata:

- generator: with output only  

- acceptor: with input only  

- transducer: with both input and output

3. \textbf{Computability}

This is the study of the problem-solving capabilities of computational models.

We can classify problems based on resources:

- Impossible problems  

- Possible with unlimited resources but impossible with limited resources  

- Possible-with-limited-resources problems

Or by time:

- Undecidable problems  

- Intractable problems  

- Tractable problems

4. \textbf{Computational Complexity}

This is the study of the efficiency of problem-solving. To unify comparison, we use an abstract model for problem execution. A Turing machine is usually used, since although simple, it has been proved to be able to simulate any problem-solving steps designed by human beings.

Problems can be classified into:

- P: Polynomial-time problems. These are problems that can be solved quickly (in polynomial time) by a normal computer.

- NP: Non-deterministic Polynomial time. These are problems where we do not know how to solve them quickly, but if someone gives us a solution, we can verify it quickly (in polynomial time).

- NP-hard: at least as hard as every NP problem.

- NP-complete: both NP and NP-hard.

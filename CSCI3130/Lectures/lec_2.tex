\section{Three Basic Concepts}
\subsection{Language}
A language is a set of strings, where a string is a sequence of symbols from the alphabet. For example, for the alphabet \(\Sigma = \{a, b\}\) we have \(\{a, ab, abba, \dots\}\).

There are several operations we can apply to strings. Consider
\[
  w = a_1 a_2 \cdots a_n, \quad v = b_1 b_2 \cdots b_m.
\]
1. concatenation

We can concatenate strings, e.g. \(wv = a_1 a_2 \cdots a_n b_1 b_2 \cdots b_m\).

2. reverse

We can reverse a string, denoted as \(w^R = a_n \cdots a_2 a_1\).

Denote the string length as \(\vert w \vert\), where the length of concatenation is \(\vert uv \vert = \vert u \vert + \vert v \vert\).

We denote a string with no letters by \(\lambda\). Therefore, \(\lambda w = w \lambda = w\).

A substring of a string is a subsequence of \textbf{consecutive} characters.

In \(w = uv\), we say \(u\) is the prefix and \(v\) is the suffix.

We write \(w^n = \underbrace{ww\cdots w}_{n}\), where \(w^0 = \lambda\).

We write \(\Sigma^{\star}\) to represent the set of all possible strings from alphabet \(\Sigma\), including \(\lambda\), and \(\Sigma^+\) to represent the set of all possible strings from \(\Sigma\) except \(\lambda\).

A language is any subset of \(\Sigma^{\star}\).

\begin{note}
  Note that \(\varnothing = \{\} \neq \{\lambda\}\).
\end{note}

A language can be infinite, e.g. \(L = \{a^n b^n : n \geq 0\}\).

For operations on languages, we adopt the usual set operations. For complement, we have \(\overline{L} = \Sigma^{\star} - L\).

We define reverse as \(L^R = \{w^R : w \in L\}\). For example, for \(L = \{a^n b^n : n \geq 0\}\), we have \(L^R = \{b^n a^n : n \geq 0\}\).

We define concatenation of languages as
\[
  L_1 L_2 = \{xy : x \in L_1,\, y \in L_2\}.
\]

We write \(L^n = \underbrace{LL\cdots L}_{n}\), where \(L^0 = \{\lambda\}\).

Kleene closure is defined as
\[
  L^{\star} = L^0 \cup L^1 \cup L^2 \cup \cdots .
\]

For example, we have
\[
  \{a, bb\}^{\star} =
  \begin{dcases}
    \lambda, \\
    a, bb, \\
    aa, abb, bba, bbbb, \\
    \cdots
  \end{dcases}
\]

Positive closure is defined as
\[
  L^{+} = L^1 \cup L^2 \cup \cdots .
\]

For example, we have
\[
  \{a, bb\}^{+} =
  \begin{dcases}
    a, bb, \\
    aa, abb, bba, bbbb, \\
    \cdots
  \end{dcases}
\]

\subsection{Grammar}
Language can be described as a system of symbols, and the grammar is the rule by which the symbols are manipulated.

\begin{definition}[Grammar]
  A grammar \(G\) is defined as a 4-tuple
  \[
    G = (V, T, S, P)
  \]
  where 

  - \(V\) is a finite set of variables,

  - \(T\) is a finite set of terminals,

  - \(S \in V\) is the start variable,

  - \(P\) is a finite set of production rules.
\end{definition}

Here, the production rules \(x \to y\) specify how the grammar transforms one string into another. If \(\gamma\) can be derived from \(\alpha\) in one step, we write \(\alpha \Rightarrow \gamma\). If it can be derived using zero or more steps, we write \(\alpha \xRightarrow{\star} \gamma\).

\begin{definition}
  Let \(G = (V, T, S, P)\) be a grammar. Then the set
  \[
    L(G) = \{w \in T^{\star} : S \xRightarrow{\star} w\}
  \]
  is the language generated by \(G\).
\end{definition}

If \(w \in L(G)\), then the sequence \(S \Rightarrow w_1 \Rightarrow w_2 \Rightarrow \cdots \Rightarrow w_n \Rightarrow w\) is a derivation of the sentence \(w\), where \(S, w_1, \cdots, w_n\) are called sentential forms.

Two grammars \(G_1, G_2\) are said to be equivalent if they generate the same language, i.e. \(L(G_1) = L(G_2)\).
